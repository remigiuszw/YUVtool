\chapter{Introduction}
\label{chap:Introduction}

Reviewing the currently available YUV viewer software we realized there is a lack of free (as in freedom), cross-platform YUV viewer application. The need for such application encouraged us to develop one.

\section{Desired features.}

We would like to include following features observed in other similar software:
\begin{itemize}
  \item{} support for configurable resolution,
  \item{} support for planar and interleaved formats,
  \item{} support for configurable bit depth (8-bit, 10-bit, etc.), configured for each plane separately,
  \item{} support for configurable plane ordering and meaning (monochrome, YUV, RGB, CMY, etc.),
  \item{} fast image rendering, scaling and rotation using hardware support (OpenGL),
  \item{} comparison of frames originating from different files,
  \item{} recognition of the video format based on the file name,
  \item{} display of pixel data in hex format,
  \item{} conversion between different formats
\end{itemize}

Apart from above we intend to implement following functionality we haven't yet observed in other applications:
\begin{itemize}
  \item{} basic YUV editing features: hex edit, frames insertion, deletion, etc.
  \item{} encoding the description of YUV format in supplementary per YUV file files,
  \item{} project and/or workspace files containing: the list of open files, positions in each file, comparisons
\end{itemize}


Features will be added gradually, beginning with the core ones. However, a design allowing easy integration of the rest of them, will be maintained throughout the entire development period.
